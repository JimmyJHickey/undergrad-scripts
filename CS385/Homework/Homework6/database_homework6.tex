\documentclass[]{article}

\begin{document}
\noindent Jimmy Hickey\\
CS 385: Applied Database Management Systems\\
22-4-16

\section*{} 
a. Compute B\textsuperscript{+}.\\
$ 
B\textsuperscript{+} = (B)\\
B\rightarrow D \indent \indent B\textsuperscript{+}=(B,D)\\
D\rightarrow A \indent \indent B\textsuperscript{+}=(A,B,D)\\
A\rightarrow BCD \indent B\textsuperscript{+}=(A,B,C,D)\\
BC\rightarrow DE \indent B\textsuperscript{+}=(A,B,C,D,E)\\
$
\\
b. Prove (using Armstrong’s axioms) that AF is a superkey.\\
$
(AF)\textsuperscript{+}\rightarrow(A) 
\indent\indent(AF)\textsuperscript{+}=(A)
\indent\indent\indent\indent\indent(Reflexitivity)\\
%
(AF)\textsuperscript{+}\rightarrow(F)
\indent\indent(AF)\textsuperscript{+}=(A,F)
\indent\indent\indent\indent(Reflexitivity)\\
%
A\rightarrow BCD
\indent\indent\indent(AF)\textsuperscript{+}=(A,B,C,D,F)
\indent(FD1)\\
%
A\rightarrow BC\rightarrow DE
\indent(AF)\textsuperscript{+}=(A,B,C,D,E,F)
\indent(Transitivity)\\
$\dfrac{num}{den}
%
R$\subseteq(AF)$\textsuperscript{+}, AF is a superkey.
\\\\
c. Compute a canonical cover for the above set of functional dependencies
F; give each step of your derivation with an explanation.\\
%
Using the union rule $B\rightarrow D$ is extraneous.\\
F'=\{$A\rightarrow BCD, BC\rightarrow DE, D \rightarrow A$\}\\
%
C is extraneous in $BC\rightarrow DE$.\\
$(B)\textsuperscript{+}=(A,B,C,D,E)$\\
F''=\{$A\rightarrow BCD, B\rightarrow DE, D \rightarrow A$\}\\
%
D is extraneous in $A\rightarrow BCD$.\\
$(A)\textsuperscript{+}=(A,B,C,D,E)$\\
F'''= $F_c$=\{$A\rightarrow BC, B\rightarrow DE, D \rightarrow A$\}
\\\\
d.\indent Give a 3NF decomposition of \textit{r} based on the canonical cover.\\
\{(A,B,C), (B,D,E), (D,A), (A,F)\}
\\\\
e.\indent Give a BCNF decomposition of \textit{r} using the original set of functional dependencies.\\
\{(A,B,C,D), (A,E,F)\} $\Rightarrow$ \{(BD), (ABC), (AEF)\}
\\\\
f.\indent Can you get the same BCNF decomposition of \textit{r} as above, using the canonical cover?\\
No you cannot.




\end{document}

